\documentclass[12pt]{article}

\begin{document}
\title{Recommendation System (Proyectos de IA-2022)}
\author{
    Reinaldo Barrera Travieso (@Reinaldo14)(@ReinaldoBarrera)\\
    Juan Jose Lopez Martinez (@jotacib3)(@juanjoselm)
}
\date{}
\maketitle

\newpage

\section*{Objetivo}

Para construir un sistema de recomendación para los clientes de productos, videos y transmisión de música, y más, con la ayuda de ANN, data mining, máquina de aprendizajey programación.

\section*{Problema}

La competencia es alta en todos los dominios, ya sea comercio electrónico o entretenimiento. Y para destacar, debes recorrer kilómetros extra. Si ofrece algo que su cliente objetivo está buscando pero no tiene las medidas para guiarlo a su tienda o recomendar sus ofertas, deja mucho dinero en efectivo sobre la mesa.

\section*{Primera Solución}
El uso de un sistema de recomendación puede atraer más visitantes a su sitio o aplicación de manera efectiva. Es posible que haya observado que las plataformas de comercio electrónico como Amazon ofrecen recomendaciones de productos que ha buscado en algún lugar de Internet. Cuando abres tu Facebook o Instagram, ves productos similares. Así es como funciona un sistema de recomendación.

Para construir este sistema, necesita un historial de navegación, comportamiento del cliente y datos implícitos. Las habilidades de minería de datos y aprendizaje automático son necesarias para producir las recomendaciones de productos más adecuadas según los intereses de los clientes. Y también necesitará programar en R, Java o Python y aprovechar las redes neuronales artificiales.

\section*{Aplicación:}
Los sistemas de recomendación encuentran aplicaciones enormes en tiendas de comercio electrónico como Amazon, eBay, servicios de transmisión de video como Netflix y YouTube, servicios de transmisión de música como Spotify y más. Ayuda a aumentar el alcance del producto, la cantidad de clientes potenciales y clientes, la visibilidad en varios canales y la rentabilidad general.

\end{document}